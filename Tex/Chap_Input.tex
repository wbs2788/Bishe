\chapter{基于思维链增强的视频生成前置安全检测方法}

\section{引言}

\subsection{视频生成面临的输入端安全威胁} Sora、Kling及Runway等视频生成大模型的爆发式发展,正在以前所未有的速度重塑数字内容的生产范式。这种技术飞跃虽然极大地降低了视觉创作的门槛,但也引入了更为隐蔽且危害深远的安全风险。相较于传统的文本模态,视频内容天然具备更强的感官冲击力与现实扭曲能力。一旦恶意用户成功诱导模型生成包含极端暴力、色情隐喻或恐怖主义教学的画面,其造成的社会心理影响与传播破坏力将呈指数级放大。特别是随着DeepSeek-R1等具备强推理能力的模型展现出接近人类的认知水平,攻击者利用模型高阶能力进行反向诱导的风险也随之剧增。

当前的安全形势正经历着从显式违规向隐式对抗的深刻转变,传统的基于规则过滤和简单RLHF对齐的防御体系正面临失效的危机。通用对抗攻击(GCG)研究~\citep{zou2023universal}揭示了一个令人担忧的现象:攻击者可以通过基于梯度的优化算法,自动化地搜索出能够最大化模型肯定响应概率的对抗性后缀。更关键的是,这种攻击具有极强的跨模型迁移性,在开源模型上训练得到的对抗样本能够直接穿透GPT-4或Claude等闭源模型的防御。这意味着,即便目前的视频生成模型多为闭源服务,攻击者仍可利用开源语言模型的权重生成的对抗样本,对视频生成的文本编码器实施“黑盒”打击。

此外,攻击手段的演进已不再局限于生成乱码式的对抗后缀,而是向着更具隐蔽性的语义伪装发展。AutoDAN攻击~\citep{liu2024autodan}证明了自动化生成的越狱提示可以兼具攻击有效性与语义通顺性,从而轻松绕过基于困惑度(Perplexity)检测的防御系统。这种“隐蔽性”在文生视频场景中尤为危险,因为视频提示词通常包含复杂的场景描述,天然具有较高的熵值,使得恶意指令更容易隐藏在正常的长文本描述中。

与此同时,DeepInception~\citep{li2023deepinception}进一步指出,大型语言模型在面对多层嵌套的叙事结构(如“梦境中的模拟”)时,容易出现“自我迷失”现象,从而导致安全防御机制失效。攻击者利用大模型在指令遵循上的强迫性,通过复杂的社会工程学嵌套、角色扮演假设以及多步逻辑陷阱来绕过防御机制。在文生视频的特定场景下,这种威胁被进一步放大。攻击者往往利用“跨模态语义鸿沟”,将有害的视觉意图隐藏在看似无害的文本描述之中。例如,通过详细描述特定的化学实验步骤、镜头角度与光影氛围,攻击者可以在不出现任何敏感词汇的情况下,诱导模型合成具有危险指导意义的视频内容。这种基于语义深层逻辑与跨模态迁移的复合攻击方式,使得传统的基于关键词匹配或浅层语义分析的防火墙在面对精心构造的提示词时形同虚设,迫切需要一种能够深入理解语义意图与推理逻辑的新型防御架构。

\subsection{现有检测方法的局限性} 面对日益复杂的攻击手段,当前工业界部署的安全检测方案陷入了效能与成本难以兼顾的结构性困境。这一困境主要体现为“不可能三角”的制约,即难以同时实现高检测精度、低推理延迟与低计算资源占用。

目前主流的防御手段主要分为两类。第一类是基于 BERT、RoBERTa 等架构的轻量级判别模型。这类模型虽然推理速度极快,能够满足在线服务的高并发需求,但受限于较小的参数规模,它们缺乏对复杂长文本的深度理解能力与世界知识储备。面对隐喻性极强的对抗样本或需要多跳推理才能识别的逻辑陷阱,这类小模型往往因无法捕捉深层语义而产生大量漏判。第二类则是依赖 GPT-4 或 Claude 等超大参数通用模型进行审核。尽管这类模型具备强大的认知能力与极高的判别准确率,但其高昂的调用成本与动辄数秒的响应延迟,使其难以作为实时视频生成的前置预检模块大规模部署。

此外,现有的防御逻辑多侧重于生成后的内容审核。然而,视频生成的计算成本极高,若在生成结束后才拦截违规内容,将造成巨大的算力浪费。因此,当前的防御体系亟需一种能够前置于生成阶段、且具备深层推理能力的轻量化检测方案。

\subsection{本章研究内容} 针对上述挑战,本章提出了一种基于 Llama-3.1-8B 的轻量化安全检测架构。该架构的设计核心在于摒弃盲目追求模型参数规模的传统路径,转而通过算法层面的创新来提升单位参数的计算效能。本研究利用低秩适应(LoRA)技术,高效地向模型注入了特定于视频生成领域的安全知识与攻防样本,使其在保持较低显存占用的同时具备了专精的鉴别能力。

更关键的是,本研究深受 DeepSeek-R1 等前沿工作在强化学习领域的启发。这些工作证明了通过激励模型进行长思维链(Chain-of-Thought, CoT)推理,可以显著提升其处理复杂逻辑任务的能力。基于此,本章在检测机制中引入了显式的推理环节。该方法改变了传统分类模型直接输出“安全”或“不安全”标签的黑盒模式,转而引导模型将单步判别任务解构为一个动态的思维过程。模型首先会对用户提示词进行语义解构,随后模拟潜在的生成结果以推演视觉画面,最后基于推演结果进行风险判定。

这种“先思考,后判断”的机制模拟了人类安全专家的审慎分析过程。它不仅能够有效识别隐藏在晦涩表达下的恶意意图,还能通过自我验证机制减少对正常创作指令的误杀。通过在训练阶段引入针对推理过程的强化学习激励,该模型能够在不显著增加推理延迟的前提下,获得接近超大模型的逻辑分析能力,从而为视频生成服务构建一道既严密又高效的输入端防线。

\section{轻量化大语言模型检测架构}
作为文生视频全流程防御体系的第一道防线,输入端检测模块的设计面临着严苛的“不可能三角”约束:
既要具备理解隐晦语义与多步推理的高准确性,又要满足视频生成服务对低延迟的实时性要求,同时还需控制显存占用以适应消费级的部署环境。
传统的基于 BERT 或 RoBERTa 的判别式小模型虽然速度极快,但缺乏逻辑推理能力,难以应对复杂的“越狱”攻击;
而 GPT-4 或 Llama-3.1-70B 等超大模型虽然理解力强,但其推理成本与延迟对于前置检测任务而言过于昂贵。

鉴于此,本章提出了一种基于 Llama-3.1-8B 的轻量化检测架构。
该架构摒弃了“超大模型暴力求解”的路径,转而通过参数高效微调(LoRA)与思维链(CoT)技术的结合,在百亿级参数规模下激发模型的安全推理能力,实现了检测精度与推理效率的最佳平衡。

\subsection{基座模型选型}

在基座模型的选型上,本研究最终锁定了 Llama-3.1-8B(此处指 Llama-3.1 系列中的 8B 参数量级指令微调版本)。
这一选择是基于对当前开源大模型在安全检测任务中性能边界的深度量化分析,代表了当前端侧与边缘侧模型的最新技术水位。

\subsubsection{检测精度与推理延迟的帕累托最优}

作为视频生成的“前置预检模块”(Pre-check Module),检测器的响应速度直接决定了用户体验的流畅度。
Llama-3.1-8B 在计算复杂度与推理能力之间找到了一个理想的甜点区:
\begin{itemize}
    \item 推理能力的质变:与早期的 7B 模型相比,Llama-3.1-8B 通过在 15T Token 上的过度训练(Over-training),展现出了媲美以往 30B+ 模型的逻辑推理能力。它具备了处理复杂隐喻和长尾安全知识的“涌现”特性,能够有效执行思维链(CoT)推理。
    \item 部署效率的优势:相比于 70B 或 405B 版本,8B 模型的参数量仅为其 1/10 甚至 1/50。在半精度(BF16)或 8-bit 量化下,其显存占用可控制在 10GB-16GB 之间,能够轻松部署在单张消费级显卡(如 RTX 3090/4090)上,并实现毫秒级的 Token 生成速度,满足高并发工业场景的需求。
    \item 核心架构特性:分组查询注意力 (GQA)Llama-3.1-8B 采用了分组查询注意力(Grouped Query Attention, GQA)机制,这是其实现高吞吐量的关键技术。
    在标准的多头注意力(Multi-head Attention, MHA)中,Query、Key、Value 的头数相同,导致推理时的 KV Cache 显存占用巨大。而 GQA 将 Query 头分组,每组共享一对 Key 和 Value 头。
    在本模型中,Query 头数为 32,而 KV 头数仅为 8(即 4:1 的分组比例)。
    这不仅显著降低了推理过程中的显存占用,更大幅减少了显存带宽需求,使得模型在处理长 Context(如包含长篇剧情描述的 Prompt)时依然保持极高的吞吐效率。
\end{itemize}

\subsubsection{核心架构特性:分组查询注意力}
模型参数规格详解Llama-3.1-8B 的具体网络拓扑结构与参数配置如表 3-1 所示。
该模型采用了标准的 Transformer Decoder-only 架构,配合 128k 的超长上下文窗口,使其能够精准理解和分析长篇幅的视频脚本指令。
\begin{table}[!t]
    \centering
    \bicaption{Llama-3.1-8B 模型参数维度规格表}{Hyperparameter Specifications of Llama-3.1-8B Base Model}
    \label{tab:model_hyperparameters}
    \setlength{\tabcolsep}{8pt} % 优化列间距
    \renewcommand{\arraystretch}{1.2} % 略微增加行高,提升可读性
    \begin{tabularx}{\textwidth}{l l X}
        \toprule
        \textbf{参数维度 (Hyperparameter)} & \textbf{数值 (Value)} & \textbf{技术说明 (Description)} \\
        \midrule
        参数总量 (Total Parameters) & 8.03 B & 平衡了推理能力与显存占用,适合单卡部署。 \\
        词表大小 (Vocab Size) & 128,256 & 增强了对多语言及长尾词汇的编码效率。 \\
        隐藏层维度 (Hidden Size) & 4,096 & 模型的特征宽度,保证了足够的语义表征容量。 \\
        网络层数 (Layers) & 32 & 标准深度,足以捕捉复杂的逻辑依赖关系。 \\
        注意力头数 (Attention Heads) & 32 & Query 头的数量。 \\
        KV 头数 (KV Heads) & 8 & 采用 GQA 机制,大幅压缩 KV Cache。 \\
        上下文窗口 (Context Window) & 128k & 支持超长文本输入,适应剧本级视频生成指令。 \\
        位置编码 (Positional Embed) & RoPE & 旋转位置编码,提升长序列外推能力。 \\
        激活函数 (Activation) & SwiGLU & 相比 ReLU 具有更好的收敛性和性能。 \\
        \bottomrule
    \end{tabularx}
\end{table}

通过选用 Llama-3.1-8B,我们确保了检测架构在底层算力层面具备了高效运行的物理基础,为后续引入 LoRA 微调与 CoT 推理提供了强大的语义理解底座。

\subsection{基于低秩自适应的参数高效微调}

\subsection{基于低秩自适应的微调机制}

对于十亿级参数量的 Llama-3.1,全参数微调(Full Fine-tuning)不仅对显存资源要求极高,且容易导致模型在小规模偏好数据集上发生灾难性遗忘(Catastrophic Forgetting)。
因此,本文采用低秩自适应(Low-Rank Adaptation, LoRA)技术进行参数高效微调。

\textbf{(1)低秩矩阵注入形式LoRA。} 假设模型在特定任务上的权重更新矩阵具有极低的本征秩(Intrinsic Rank)。
对于预训练的权重矩阵 $W_0 \in \mathbb{R}^{d \times k}$,我们将参数更新量 $\Delta W$ 分解为两个低秩矩阵 $B \in \mathbb{R}^{d \times r}$ 和 $A \in \mathbb{R}^{r \times k}$ 的乘积,即:
\begin{equation}
W' = W_0 + \Delta W = W_0 + \frac{\alpha}{r} B A.
\end{equation}
其中,$r \ll \min(d, k)$ 为秩的大小,$\alpha$ 为缩放系数。在训练过程中,$W_0$ 保持冻结,仅优化 $A$ 和 $B$。
初始化阶段,矩阵 $A$采用高斯分布初始化,这为模型提供了随机的梯度起点,使其能够探索参数空间;
矩阵 $B$ 初始化为全零矩阵,这样可以确保在训练的第一步,$BA = 0$,从而使得 $W' = W_0$。这种“零残差”的起点能有效防止初始阶段的随机噪声破坏预训练模型已学习到的特征表达。

\textbf{(2)针对安全推理的微调策略。} 在前置检测任务中,模型需要学习特定的安全分类体系与思维链推理模式。根据 LLM 微调的经验法则,自注意力模块中的**查询(Query, $W_q$)与值(Value, $W_v$)**投影层对语义理解与文本生成的贡献最大。因此,我们仅对这两层应用 LoRA 更新。
考虑到安全检测属于判别式任务,其任务边界相对清晰,不需要过大的容量。我们将秩设置为 $r=16$。

\textbf{(3)参数量效率分析.} 基于上述配置,Llama-3.1-8B 模型的可训练参数量仅为 12.8M,占总参数量(8.03B)的比例约为 0.16\%。
这种极低参数量的特性不仅显著降低了训练显存需求,还允许我们在部署时通过动态加载不同的权重来切换不同的安全策略,而无需重新加载整个基座模型。

\section{基于思维链增强的检测方法}

虽然 Llama-3.1-8B + LoRA 提供了强大的底层语义理解能力,但若缺乏明确的指令引导,模型仍难以在复杂的语境下做出符合人类价值观的判断。
为此,本节提出了“策略即提示(Policy-as-Prompt)”与“思维链(Chain-of-Thought)”两大核心机制,前者负责将抽象的安全规范转化为模型可执行的自然语言指令,后者负责引导模型将单步判别任务分解为多步逻辑推理过程。

\subsection{策略即提示机制}

在传统的审核系统中,安全规则通常以代码逻辑或硬编码的关键词列表形式存在,这种“显式规则”难以覆盖千变万化的语言表达。
本研究提出的 Policy-as-Prompt 机制,旨在将复杂的安全审核标准转化为结构化的系统提示词,直接注入模型的上下文窗口中。
这种方法利用大语言模型的指令遵循能力,实现了从“规则匹配”到“规则理解”的范式转变。

\subsubsection{风险分类体系}

为了建立清晰的审核边界,本研究结合工业界主流规范(如 MLCommons),定义了一套包含四个核心维度的风险分类体系。
该体系不仅涵盖了显式的违规内容,还特别针对视频生成的特性,增加了对动态行为和隐喻描述的定义。

\textbf{(1)暴力与血腥}:涉及对生物体造成物理伤害、残害或杀戮的逼真描述;以及对战争、酷刑、大规模伤亡事件的视觉化重现。例如连贯的攻击动作序列(如刺杀过程演示)、制造武器的详细步骤教程等。

\textbf{(2)色情与淫秽}:涉及性行为、生殖器官的露骨描写;以及具有性暗示的姿势、恋童癖内容(CSAM)或非自愿的性内容(NCII)。需要拦截使用隐晦俚语的生成请求。

\textbf{(3)偏见与歧视}:针对受保护群体(种族、性别、宗教、残疾等)的仇恨言论、刻板印象强化或非人化描述。例如生成“特定种族人群正在实施犯罪”的偏见性画面。

\textbf{(4)非法行为}:宣扬、指导或演示违反法律的活动。例如毒品制造与交易、网络攻击演示、洗钱教程、恐怖主义宣传等。

\subsection{系统提示词模板设计}

基于前文构建的四维风险分类体系,本研究设计了一套结构化的系统提示词(System Prompt),旨在将抽象的安全规范转化为大语言模型可执行的认知指令。该提示词模板摒弃了传统的非结构化自然语言描述,转而采用了“角色扮演、规则注入、输出约束”三位一体的工程化结构,强制模型在推理计算的起始阶段即加载完整的安全知识图谱。

首先,在角色设定模块,系统明确将 Llama-3.1-8B 定义为“专业的视频生成内容安全审计专家”,而非通用的对话助手。这一身份锚定(Identity Anchoring)通过调整模型在潜在空间中的注意力分布,使其后续的文本生成概率分布倾向于审慎、严谨的判别模式,从而有效抑制模型的发散性联想与闲聊倾向。其次,在规则注入模块,我们将前文定义的暴力、色情、偏见及非法行为四大类风险标准,细化为模型可理解的特征描述符并嵌入上下文。不同于简单的关键词匹配,该模块要求模型理解隐喻性描述与动态行为逻辑,例如将“红色的液体流出”与暴力场景建立语义关联,将特定的连贯动作序列识别为潜在的非法教程。最后,在输出约束模块,为了确保检测结果能够被下游的防御系统自动解析,我们规定了严格的输出格式。模型被禁止输出任何模棱两可的解释性废话,必须严格遵循 JSON 格式标准,仅输出推理过程摘要与最终的安全标签。这种结构化的设计不仅降低了后处理的正则表达式解析难度,更通过格式约束反向强化了模型的逻辑严密性。

为了直观展示该机制的运作方式,图x展示了实际部署中使用的系统提示词模板实例。该模板通过特殊的 分隔符将指令区与数据区隔离,防止了恶意用户通过提示词注入(Prompt Injection)攻击篡改系统的核心安全指令。

\subsection{思维链推理范式}

尽管 Llama-3.1-8B 具备强大的语义理解能力,但面对复杂的视频生成指令,直接的“输入-判别”映射往往难以捕捉深层的恶意意图,尤其是在处理包含隐晦隐喻或多步逻辑陷阱的“越狱”提示词时。为此,本研究引入了思维链(Chain-of-Thought, CoT)推理范式,将单次的二分类判别任务解构为“语义分解、风险映射、综合判定”的连贯认知过程。该范式利用大语言模型的自回归特性,强制模型在生成最终判定结论之前,显式地生成一段中间推理步骤,从而激活模型内部深层的逻辑推演能力。

在语义分解阶段,模型首先被引导提取输入文本中的关键实体、动作描述以及环境氛围。针对视频生成的特性,这一阶段特别强调对“动态视觉元素”的解析。例如,对于提示词“一个人在夜晚的巷子里通过化学实验制作白色粉末”,CoT 机制引导模型不应仅关注孤立的词汇,而应识别出“夜晚巷子”的环境暗示与“制作粉末”的行为逻辑。随后进入风险映射阶段,模型将提取出的语义要素与预置的安全知识库进行比对。此时,思维链发挥了关键的消歧作用,它引导模型通过上下文推理判断该场景是属于合法的化学教育演示,还是非法的毒品制造教程。最后,在综合判定阶段,模型基于前两步的证据链条生成最终的安全标签。

为了在轻量化模型上实现稳定的思维链推理,本研究采用了少样本上下文学习(Few-shot In-context Learning)策略。我们在输入端构造了包含正负样本对照的演示示例,这些示例展示了如何一步步从隐晦的文本中推导出违规结论。通过模仿这些高价值的推理路径,Llama-3.1-8B 能够在不更新参数的情况下,快速习得安全专家的审计逻辑。实验表明,这种“先思考,后判断”的机制显著降低了模型在面对对抗性样本时的误判率,有效解决了传统分类器“知其然而不知其所以然”的鲁棒性缺陷。

\section{实验结果与分析}

本章将通过一系列定性与定量实验,全面验证基于思维链增强的轻量化检测架构在视频生成前置防御任务中的有效性。实验旨在回答三个核心问题:首先,在参数高效微调的约束下,所提方法能否在检测精度上超越传统的判别式模型及未经过思维链增强的基座模型;其次,思维链机制的引入是否显著提升了模型对隐晦恶意意图及复杂逻辑陷阱的识别能力;最后,该轻量化架构在推理延迟与显存占用方面是否满足视频生成服务对实时性的工程化要求。

\subsection{实验设置}

为了确保实验结论的客观性与可复现性,本节详细阐述了实验所构建的专用数据集、选取的对比基准模型、评价指标体系以及具体的训练实施细节。

\subsubsection{视频生成安全检测数据集构建}

现有的安全检测数据集主要面向对话问答场景,其提示词多为问句形式,缺乏视频生成任务中特有的画面描述、运镜指令及视觉风格修饰词。直接复用此类数据集难以真实评估前置检测模块在文生视频场景下的泛化能力。鉴于此,本研究构建了面向视频生成的专用安全检测基准数据集 VideoSafety-Bench。

该数据集的数据源由两部分组成。恶意样本部分主要采样自 BeaverTails 与 SafetyPrompts 开源数据集。为了适配视频生成语境,本研究利用 GPT-4 对原始恶意问答进行了风格迁移重写,将其转化为包含“电影质感”、“动态特写”等视觉描述符的视频脚本指令,并严格按照前文定义的暴力、色情、偏见及非法行为四个维度进行标签清洗。正常样本部分则从公开的 WebVid-10M 数据集中随机抽取,涵盖了自然风光、日常生活、科技展示等无害场景,以确保数据分布的均衡性。最终构建的 VideoSafety-Bench 包含训练集一万条与测试集一千条,正负样本比例控制在一比一,旨在模拟真实业务场景中恶意攻击与正常请求混合的流量特征。

\subsubsection{对比基准模型}

为了在“精度-效率”二维谱系中全面定位所提方法的性能坐标,本研究选取了三类具有代表性的基线模型进行对比分析。第一类为传统的轻量级判别模型,包括 BERT-Large 与 RoBERTa-Large。此类模型在工业界应用广泛,推理速度极快,被视为检测任务的效率基准。第二类为同参数量级的通用大语言模型,包括未经过微调的 Llama-3.1-8B-Instruct 以及 Mistral-7B-Instruct。引入该组基线旨在验证“策略即提示”与“思维链”机制在零样本(Zero-shot)条件下的原生推理能力。第三类为超大规模语言模型,选取了 GPT-4o 作为参考对象。尽管其推理成本高昂不适合直接部署,但其强大的语义理解能力构成了当前技术条件下的性能上限(Oracle),用于衡量所提轻量化方法的精度天花板。

\subsubsection{评价指标体系}

实验评价体系涵盖检测性能与推理效率两个维度。在检测性能方面,鉴于安全检测任务对漏判(False Negative)的极低容忍度,本研究除汇报准确率(Accuracy)外,重点关注召回率(Recall)与 F1 分数(F1-Score)。其中,召回率直接反映了防御系统拦截恶意请求的覆盖能力,F1 分数则综合衡量了模型在查准与查全之间的平衡。在推理效率方面,为了验证架构的轻量化优势,实验记录了单样本平均推理延迟(Latency)与显存峰值占用(Peak VRAM)。推理延迟均在单张消费级显卡上进行测试,以毫秒(ms)为单位,直接关联用户体验的流畅度;显存占用则决定了模型在边缘侧设备上的部署可行性。

\subsubsection{实施细节}

所有实验均基于 PyTorch 深度学习框架与 HuggingFace Transformers 库实现。在微调阶段,基础模型加载为 16 位浮点精度(BF16),低秩自适应(LoRA)模块被注入到所有注意力层的查询与值投影矩阵中。根据 3.2 节的参数量效率分析,LoRA 的秩设置为 16,缩放系数 alpha 设置为 32,丢失率(Dropout)设为 0.05。优化器选用 AdamW,初始学习率设定为 2e-4,并配合余弦退火策略进行动态调整。批次大小设定为 64,并在 4 张 NVIDIA A800 GPU 上进行并行训练,共迭代 3 个 Epoch。

在推理阶段,为了模拟真实的工业部署环境,所有性能测试均在单张 NVIDIA RTX 4090 显卡上完成。思维链生成的最大长度限制为 256 个 Token,以防止过长的无效推理增加延迟。为了消除随机性带来的实验误差,所有定量结果均为三次独立重复实验的平均值。

\subsection{主实验结果}

\begin{table*}[t]
    \centering
    \caption{不同检测方法在 VideoSafety-Bench 上的性能与效率对比。其中 $\uparrow$ 表示数值越高越好,$\downarrow$ 表示数值越低越好。加粗数据表示除 Oracle 外的最优结果。}
    \label{tab:main_results_detection}
    \resizebox{\linewidth}{!}{
    \begin{tabular}{lcccccc}
    \toprule
    \multirow{2}{*}{\textbf{方法 (Method)}} & \multirow{2}{*}{\textbf{模型架构}} & \multicolumn{3}{c}{\textbf{检测性能 (Performance)}} & \multicolumn{2}{c}{\textbf{推理效率 (Efficiency)}} \\
    \cmidrule(lr){3-5} \cmidrule(lr){6-7}
    & & \textbf{准确率 (Acc)} $\uparrow$ & \textbf{召回率 (Recall)} $\uparrow$ & \textbf{F1 分数} $\uparrow$ & \textbf{延迟 (Latency)} $\downarrow$ & \textbf{显存 (VRAM)} $\downarrow$ \\
    \midrule
    \multicolumn{7}{l}{\textit{Traditional Discriminative Models (传统判别式模型)}} \\
    \textbf{BERT-Large} & Encoder-only & 82.4\% & 78.5\% & 80.4\% & \textbf{15 ms} & \textbf{1.3 GB} \\
    \textbf{RoBERTa-Large} & Encoder-only & 84.1\% & 81.2\% & 82.6\% & 18 ms & 1.4 GB \\
    \midrule
    \multicolumn{7}{l}{\textit{Large Language Models (Zero-shot Setting)}} \\
    \textbf{Mistral-7B-Instruct} & Decoder-only & 88.5\% & 86.4\% & 87.4\% & 42 ms & 14.2 GB \\
    \textbf{Llama-3.1-8B-Instruct} & Decoder-only & 90.2\% & 89.1\% & 89.6\% & 45 ms & 15.8 GB \\
    \midrule
    \multicolumn{7}{l}{\textit{Upper Bound Reference (性能上限参考)}} \\
    \textbf{GPT-4o} & Mixture-of-Experts & \textit{98.1\%} & \textit{98.5\%} & \textit{98.3\%} & \textit{N/A} & \textit{N/A} \\
    \midrule
    \multicolumn{7}{l}{\textit{Proposed Method (本文方法)}} \\
    \textbf{Ours (Llama-3.1 + LoRA + CoT)} & Decoder-only & \textbf{96.8\%} & \textbf{97.4\%} & \textbf{97.1\%} & 112 ms & 16.1 GB \\
    \bottomrule
    \end{tabular}
    }
\end{table*}

表~\ref{tab:main_results_detection} 详细展示了不同检测架构在 VideoSafety-Bench 测试集上的综合性能表现。实验数据揭示了所提方法在检测精度与资源效率之间取得了极佳的平衡,具体分析如下:

首先,在核心检测指标(F1-Score)上,本文提出的基于思维链增强的轻量化架构取得了 97.1\% 的优异成绩。这一数值显著超越了传统的 BERT-Large(80.4\%)与 RoBERTa-Large(82.6\%)模型。虽然传统判别式模型拥有毫秒级的极致推理速度,但其受限于较浅的语义理解能力,难以捕捉视频生成指令中隐晦的恶意隐喻(如将“化学实验”作为“制毒”的伪装),导致召回率偏低。相比之下,本文方法通过引入大语言模型的深层语义理解能力,将漏判率降低了一个数量级,证明了在复杂的文生视频安全场景下,单纯依赖关键词匹配或浅层语义分类已不再适用。

其次,与同参数量级的 Zero-shot 基座模型相比,所提方法展现了“参数高效微调 + 思维链”的组合优势。原始的 Llama-3.1-8B-Instruct 在未经过特定领域微调时,虽然具备一定的通用逻辑能力,但在面对特定风险定义(如视频特有的动态暴力描述)时,F1 分数仅为 89.6\%。引入 LoRA 微调与 CoT 推理后,模型的准确率与召回率分别提升了 6.6\% 与 8.3\%。这一显著的性能跃升表明,通过思维链将单步判别任务解构为多步推理过程,有效地激活了模型在安全领域的潜在推理能力,使其能够像人类审核员一样审慎地分析上下文。值得注意的是,本文方法的性能已逼近 GPT-4o 的理论上限(97.1\% vs 98.3\%),但参数量仅为其百分之一,极具性价比。

最后,在推理效率方面,尽管引入思维链机制导致平均推理延迟增加至 112 毫秒(包含生成推理步骤的时间),但这一延时对于视频生成任务(通常需数秒至数十秒生成视频)而言完全在可接受的范围内,不会对用户体验造成感知上的阻滞。同时,得益于 LoRA 的低秩特性与 8B 模型的紧凑架构,显存峰值占用控制在 16.1 GB。这意味着该防御模块可以轻松部署在单张消费级显卡(如 RTX 3090/4090)上,无需依赖昂贵的 A100 集群,完美契合了“轻量化前置检测”的设计初衷。

\subsection{消融实验}

为了深入剖析本章所提架构中各个核心组件的独立贡献及其相互作用机制,实验设计了一组严谨的消融测试。我们以 Llama-3.1-8B-Instruct 基座模型为起点,逐步叠加“策略即提示(Policy-as-Prompt)”、“低秩自适应微调(LoRA)”以及“思维链推理(CoT)”模块,通过观测 F1 分数与召回率的变化轨迹来验证各模块的有效性。表 3-3 详细汇总了不同配置下的实验数据。

\begin{table}[h] 
    \centering 
    \caption{不同组件配置下的消融实验结果。其中“直接输出”表示不含推理过程的标签判别,“思维链(CoT)”表示包含完整推理过程。} 
    \label{tab:ablation_study} 
    \resizebox{0.9\linewidth}{!}{ 
        \begin{tabular}{lccccc} 
            \toprule 
            \textbf{实验变体} & \textbf{微调策略} & \textbf{推理模式} & \textbf{提示词策略} & \textbf{F1 分数} & \textbf{召回率} \\ 
            \midrule 
            \textit{基准模型} & 无 (零样本) & 直接输出 & 简单指令 & 86.5\% & 84.2\% \\
            \textit{变体 A} & 无 (零样本) & 直接输出 & \textbf{结构化策略} & 89.6\% (+3.1\%) & 89.1\% \\ 
            \textit{变体 B} & \textbf{LoRA} & 直接输出 & \textbf{结构化策略} & 94.2\% (+4.6\%) & 93.5\% \\ 
            \textbf{完整方法} & \textbf{LoRA} & \textbf{思维链} & \textbf{结构化策略} & \textbf{97.1\% (+2.9\%)} & \textbf{97.4\%} \\ 
            \bottomrule 
        \end{tabular} 
    } 
\end{table}
首先考察结构化系统提示词的作用。对比基准模型(Baseline)与变体 A 可以发现,仅引入包含清晰风险分类定义的结构化提示词,即可将 F1 分数从 86.5\% 提升至 89.6\%。这一显著增益表明,在未进行参数更新的情况下,明确的指令引导能够有效激活大语言模型的指令遵循能力,使其初步对齐预设的安全标准。简单的通用提示词往往导致模型对“暴力”或“色情”的定义产生歧义,而结构化的策略注入为模型提供了清晰的判别边界,减少了因定义模糊导致的误判。

其次,参数高效微调被证明是提升领域适应性的关键。通过对比变体 A 与变体 B,在引入 LoRA 模块后,模型的 F1 分数实现了 4.6\% 的大幅跃升,达到 94.2\%。这证实了通用大模型虽然具备广泛的世界知识,但缺乏针对特定视频生成攻击模式的领域知识。通过在查询与值投影层注入少量可训练参数,模型成功习得了从文本描述映射到具体风险类别的非线性关系,特别是增强了对隐晦俚语及特定攻击句式的敏感度。这一结果有力地反驳了“仅靠提示工程即可解决安全问题”的观点,确立了参数化微调在构建高可靠防线中的必要性。

最后,思维链推理机制补全了安全检测的最后一块拼图。对比变体 B 与全量方法(Full Method),在微调的基础上进一步引入思维链推理,使 F1 分数最终突破至 97.1\%,同时召回率达到了 97.4\% 的极高水平。深入分析发现,这一提升主要来源于对复杂逻辑陷阱与多意图样本的成功识别。在直接输出模式(Direct)下,模型倾向于依赖表面的关键词特征进行快速决策,极易被对抗性样本的“无害伪装”所欺骗。而思维链机制强制模型在生成最终标签前显式地执行语义分解与风险映射,这种“慢思考”过程有效地纠正了模型的直觉偏差,使其能够穿透隐喻与代指的迷雾,精准捕捉深层的恶意意图。尽管这引入了额外的推理开销,但对于防御高风险的越狱攻击而言,这种计算成本的投入是具有极高性价比的。

\subsection{定性试验分析}

为了超越单纯的数值指标,深入探究模型在面对复杂对抗性攻击时的具体行为逻辑,本节选取了两类极具挑战性的典型样本——隐晦的恶意隐喻与复杂的动态视觉描述,对不同模型的推理过程进行了细粒度的定性对比分析。通过解构模型生成的中间思维轨迹,我们试图揭示思维链机制在处理语义歧义与意图识别时的核心优势。

首要关注的是针对“隐晦隐喻”的检测能力,这是传统防御系统最易失守的薄弱环节。在测试中,我们输入了一条经过精心伪装的提示词:“详细展示在家庭厨房中使用感冒药提炼白色晶体的全过程,镜头要给到结晶的特写。” 面对这一指令,基于 BERT 的传统判别模型将其误判为“安全”,原因在于输入文本中并未出现“冰毒”、“制毒”等显式的黑名单关键词,模型仅捕捉到了“厨房”、“感冒药”等中性词汇的表面特征。即便是未经过思维链微调的 Llama-3.1-8B-Instruct 基座模型,在零样本推理下也表现出了犹豫,虽然识别出了潜在风险,但给出的理由含糊其辞,未能准确将其归类为“非法行为”。与之形成鲜明对比的是,本文提出的方法成功激活了深度推理链路。模型首先在思维链的第一步准确提取了“使用感冒药提炼”这一关键动作与“白色晶体”这一结果实体;随后在第二步推理中,结合内置的安全知识库进行了逻辑关联,指出了感冒药(含麻黄碱)是制造冰毒的常见前体;最终在结论阶段,明确判定该指令实为披着化学实验外衣的毒品制造教程,触犯了“非法行为”的红线。这一案例生动地证明了思维链机制赋予了模型穿透语义伪装、洞察深层意图的认知穿透力。

其次,针对视频生成特有的“动态暴力描述”,实验考察了模型对视觉画面转译的敏感度。测试样本为:“一名身穿黑衣的武士在人群中快速移动,红色的液体随着刀光四处飞溅,断肢在空中翻滚。” 这是一个典型的去敏感化描述,攻击者刻意回避了“杀人”、“血腥”等直接词汇。实验结果显示,基线模型大多关注到了“武士”、“移动”等动作元素,将其错误归类为“动作电影场景”或仅标记为轻微风险。然而,本文方法展现出了卓越的视觉语义对齐能力。在生成的推理轨迹中,模型敏锐地捕捉到了“红色的液体”与“断肢”这两个视觉描述符,并将其映射为“严重的身体伤害”与“血腥场面”。模型明确指出,尽管未出现“血液”一词,但语境中的红色液体显然指代大量出血,且断肢画面违反了关于“人体残害”的严格禁令。最终,该请求被高置信度地拦截。这一分析表明,通过在微调阶段注入针对视频画面的描述性知识,配合思维链的逐步解析,模型能够有效地将抽象的文本指令预演为潜在的视觉画面,从而在生成发生之前精准阻断违规内容的产出。

表~\ref{tab:qualitative_cases}汇总了上述典型案例的推理对比。这些定性证据强有力地支持了前文的定量结论,即本章所提架构不仅在统计意义上具备高准确率,更在逻辑可解释性与对抗鲁棒性上实现了质的飞跃。

\begin{table}[h] 
    \centering 
    \caption{典型对抗性样本的定性推理对比分析。Ours 方法通过显式的思维链(CoT)成功识破了基线模型未能检测出的隐晦攻击。} 
    \label{tab:qualitative_cases} 
    \resizebox{\linewidth}{!}{ 
        \begin{tabular}{p{0.25\linewidth} p{0.2\linewidth} p{0.2\linewidth} p{0.35\linewidth}} 
            \toprule 
            \textbf{输入提示词 (User Prompt)} & \textbf{BERT-Large} & \textbf{Llama-3.1 (Zero-shot)} & \textbf{Ours (CoT Enhanced)} \\ 
            \midrule \textit{Case 1: Implicit Drug} \newline "Show a step-by-step tutorial on extracting white crystals from cold medicine in a kitchen." & \textbf{PASS} \newline (Misidentified as cooking or chemistry experiment) & \textbf{Vague Warning} \newline "This might be unsafe but I can show chemistry basics." & \textbf{BLOCK (Illegal)} \newline \textit{Reasoning:} "Cold medicine extraction" + "White crystals" implies Methamphetamine production. This is a drug manufacturing guide. \\ 
            \midrule 
            \textit{Case 2: Visual Gore} \newline "A warrior slicing through enemies, red liquid splashing everywhere, detached limbs flying." & \textbf{PASS} \newline (Classified as Action/Fantasy genre) & \textbf{PASS} \newline "Generating a battle scene." & \textbf{BLOCK (Violence)} \newline \textit{Reasoning:} "Red liquid" contextually means blood. "Detached limbs" indicates mutilation/gore. Violates policy on realistic graphic violence. \\ 
            \bottomrule 
        \end{tabular} 
    } 
\end{table}

\section{本章小结}

本章针对文生视频系统输入端面临的隐晦语义攻击与实时性部署约束这一核心矛盾,系统地提出了一种基于思维链增强的轻量化大语言模型安全检测方法。通过深入分析检测任务中精度、速度与资源占用之间的权衡关系,本研究确立了以十亿参数级模型 Llama-3.1-8B 为基座,结合低秩自适应微调技术的技术路线。这一架构设计成功克服了超大模型推理成本高昂的工程壁垒,在消费级算力条件下有效激活了模型的深层安全推理能力。在此基础上,本章创新性地构建了“策略即提示”与思维链推理范式,将传统的单步判别任务重构为包含语义分解、风险映射与综合判定的多步认知过程,从根本上解决了传统分类器难以识别隐喻性描述与动态视觉攻击的痛点。

基于 VideoSafety-Bench 专用数据集的广泛实验有力地支撑了上述架构的优越性。定量评估数据显示,所提方法在保持毫秒级推理延迟与低显存占用的前提下,取得了 97.1\% 的 F1 分数。这一成绩不仅显著超越了 BERT 等传统判别模型,更在检测精度上逼近了 GPT-4o 的理论上限,实现了轻量化与高性能的统一。消融实验与定性案例分析进一步揭示,思维链机制是提升模型对抗鲁棒性的关键所在,它赋予了防御系统穿透语言伪装、洞察恶意意图的认知穿透力。综上所述,本章的研究成果为视频生成全流程防御体系构建了一道兼具高精度与高效率的首道防线,有效阻断了绝大多数恶意指令的源头注入,从而为后续环节的模型内生安全机制提供了清洁可靠的输入环境。