\section{本文工作总结}

随着以 Sora、Wan 为代表的视频生成大模型技术的爆发式增长,其伴生的内容安全风险已成为制约产业健康发展的关键瓶颈。针对现有防御手段在处理复杂动态视频内容时存在的检测滞后、概念纠缠及防御过载等问题,本文围绕“文生视频模型安全的全流程防御”这一核心命题,展开了系统性的研究工作。通过理论分析、算法设计、数据构建与系统实现,本文构建了一套由外及内、动静结合的安全防御体系,主要工作总结如下:

\subsection{视频生成全流程防御体系的构建}

针对当前文生视频技术面临的严峻安全挑战,特别是隐晦语义注入攻击与显式暴力内容生成的泛滥,本文系统性地构建了一套全流程安全防御体系。该体系创新性地确立了“双重防御、协同联动”的设计思想,打破了传统单一防御手段的局限性。在架构上,本研究将基于大语言模型的输入端前置阻断机制与基于扩散模型的模型端内生安全机制进行了有机融合。前者负责在指令输入源头过滤高风险的显式攻击,后者则在生成过程中对漏网的隐蔽风险进行静默化处理。通过这种由外及内的纵深防御策略,本文实现了从恶意指令拦截到视频生成去风险化的全链路闭环,有效解决了现有防御方案在面对复杂攻击时存在的漏判率高及误伤正常创作等痛点问题,为文生视频服务的工业化部署提供了可靠的安全保障方案。

\subsection{基于思维链增强的轻量化前置检测方法}

在输入端防御环节,为了在有限的计算资源下实现对隐晦恶意意图的精准识别,本文提出了一种基于思维链增强的轻量化前置检测方法。该方法以十亿参数级的 Llama-3.1-8B 为基座,通过低秩自适应(LoRA)微调技术注入了领域特定的安全知识。更为关键的是,本研究引入了思维链(Chain-of-Thought)推理范式,将传统的单步分类任务重构为包含语义分解、风险映射与综合判定的多步逻辑推演过程,成功激发了轻量级模型对隐喻性描述及复杂逻辑陷阱的深层理解能力。实验结果表明,该方法在保持毫秒级流式响应速度的同时,显著提升了对抗性样本的检测成功率,其核心安全指标已逼近 GPT-4 等超大规模模型的性能水平,为工业界在边缘侧部署高效能的安全哨兵提供了可行的技术路径。

\subsection{基于潜空间偏好优化的概念擦除机制与 VCE-24K 数据集}

在模型内生安全环节,针对视频生成过程中难以剥离特定风险概念的难题,本文提出了一种基于潜空间偏好优化的概念擦除算法(Latent-DPO)。该算法利用潜空间的分支演化策略,在不破坏视频时空一致性的前提下,实现了对暴力行为、侵权形象等特定负面概念的精准剥离与抑制。为了支撑这一算法的训练与评估,本文构建并开源了首个面向视频生成的高质量、成对概念擦除基准数据集 VCE-24K,有效填补了该研究领域的数据空白。通过在该数据集上的广泛实验,证明了 Latent-DPO 能够在保持视频基础生成质量与物理规律连贯性的同时,高效地实现模型输出与人类安全偏好的对齐,为构建内生安全的视频生成大模型奠定了坚实的算法基础。

\subsection{存在的不足}

尽管本文构建的全流程防御体系在实验环境下展现出了优异的防御性能,但受限于当前大模型技术的内在机理以及攻防对抗的演进速度,本研究在语义解耦的精细度与跨域防御的鲁棒性方面仍存在一定的局限性,尚需在未来的工作中进一步探索与完善。

\subsubsection{语义纠缠与模糊概念的边界效应}

首要的局限性体现在概念擦除算法对多义词及模糊概念处理时的边界效应问题。尽管 Latent-DPO 算法能够有效抑制高置信度的恶意概念,但在面对具有“语义纠缠”特性的复杂指令时,模型难以在潜空间中实现完美的特征正交化剥离。具体而言,对于存在多义性的名词,模型容易出现过度擦除或误伤现象。以“枪支(Gun)”这一概念为例,当防御目标设定为擦除杀伤性武器时,由于视觉特征的相似性,模型可能会错误地抑制“玩具水枪”或“加油枪”等在语义上无害但在视觉上同构的物体生成,导致正常场景的生成质量受损。此外,在处理依赖上下文界定的模糊概念时,当前防御机制的判别稳定性仍有提升空间。例如在界定“色情低俗”与“人体艺术”的边界时,由于缺乏对文化语境与审美意图的深层理解,模型偶尔会将古典油画风格的裸露描写误判为违规内容进行擦除,这种对高层语义理解的偏差反映了当前对齐算法在处理主观性概念时的内在瓶颈。

\subsubsection{多语言与跨模态攻击的鲁棒性瓶颈}

次要的局限性在于防御体系在应对低资源语言及跨模态输入时的鲁棒性瓶颈。目前的输入端检测模块主要基于中英文主导的指令数据进行微调,尽管 Llama-3.1 基座模型具备一定的多语言能力,但在面对小语种输入或经过复杂机器翻译伪装的“翻译攻击”时,其意图识别的准确率会出现衰减。攻击者可能利用模型在稀缺语料上的对齐缺陷,通过将恶意指令转译为冷门语言绕过前置检测,而后续的视频生成模型若对该语言具备一定的理解能力,则可能导致防御穿透。此外,随着视频生成技术向“图生视频(Image-to-Video)”范式演进,本研究当前主要聚焦于文本指令防御的架构显露出覆盖盲区。对于通过输入包含隐晦恶意信息的参考图像来诱导视频生成的视觉注入攻击,当前基于纯文本分析的防御体系尚无法建立有效的感知与阻断机制,这构成了全流程防御中亟待补全的一环。

\subsection{未来工作展望}

基于本文在文生视频全流程防御体系方面的探索与实践,未来的研究工作将致力于突破现有的技术边界,从多概念协同擦除与动态自适应防御两个维度进一步提升系统的鲁棒性与泛化能力。

\subsubsection{多概念并发擦除与模块化解耦}

当前的研究主要聚焦于对单一特定风险概念的精准剥离,但在实际且复杂的生成场景中,模型往往需要同时应对暴力、色情、侵权等多种混合风险的挑战。未来的研究将致力于向大规模多概念协同擦除的方向演进,重点解决在同时抑制多个概念时可能引发的参数干扰或灾难性遗忘问题。为此,拟探索基于“即插即用”架构的模块化安全专家适配器机制。该机制旨在训练一组相互解耦的安全低秩适配器,每个适配器专职负责擦除一类特定的风险概念。在推理阶段,系统允许用户根据实际业务需求动态组合并加载不同的安全模块,例如同时启用“反暴力”与“反侵权”适配器。这种模块化解耦的设计不仅能够避免因单一模型容量有限导致的擦除性能瓶颈,更使得防御系统具备了极高的灵活性与可扩展性,无需对庞大的基座模型进行昂贵的全量重训练即可应对新增的安全需求。

\subsubsection{对抗训练与动态防御机制}

考虑到安全攻防具有高度的动态演化特性,静态的检测模型难以长期抵御不断推陈出新的攻击手段。未来的工作计划引入对抗训练范式,以提升前置检测器对未知攻击的泛化能力。具体而言,将构建一个自动化的攻击生成代理,专门负责生成能够绕过当前防御体系的对抗性提示词。这些高难度的对抗样本将被实时标注并回流至训练集中,用于对检测模型进行持续的迭代微调。在此基础上,拟进一步建立“红蓝对抗”自动化评估闭环。通过红方攻击代理与蓝方防御模型的持续博弈,模拟真实的攻防环境。这种机制将推动防御系统实现从被动拦截向主动进化的范式转变,最终构建出一套能够随着攻击手段演变而自适应迭代的内生安全防御系统。
